% Options for packages loaded elsewhere
\PassOptionsToPackage{unicode}{hyperref}
\PassOptionsToPackage{hyphens}{url}
\PassOptionsToPackage{dvipsnames,svgnames,x11names}{xcolor}
%
\documentclass[
  letterpaper,
  DIV=11,
  numbers=noendperiod]{scrartcl}

\usepackage{amsmath,amssymb}
\usepackage{iftex}
\ifPDFTeX
  \usepackage[T1]{fontenc}
  \usepackage[utf8]{inputenc}
  \usepackage{textcomp} % provide euro and other symbols
\else % if luatex or xetex
  \usepackage{unicode-math}
  \defaultfontfeatures{Scale=MatchLowercase}
  \defaultfontfeatures[\rmfamily]{Ligatures=TeX,Scale=1}
\fi
\usepackage{lmodern}
\ifPDFTeX\else  
    % xetex/luatex font selection
\fi
% Use upquote if available, for straight quotes in verbatim environments
\IfFileExists{upquote.sty}{\usepackage{upquote}}{}
\IfFileExists{microtype.sty}{% use microtype if available
  \usepackage[]{microtype}
  \UseMicrotypeSet[protrusion]{basicmath} % disable protrusion for tt fonts
}{}
\makeatletter
\@ifundefined{KOMAClassName}{% if non-KOMA class
  \IfFileExists{parskip.sty}{%
    \usepackage{parskip}
  }{% else
    \setlength{\parindent}{0pt}
    \setlength{\parskip}{6pt plus 2pt minus 1pt}}
}{% if KOMA class
  \KOMAoptions{parskip=half}}
\makeatother
\usepackage{xcolor}
\setlength{\emergencystretch}{3em} % prevent overfull lines
\setcounter{secnumdepth}{-\maxdimen} % remove section numbering
% Make \paragraph and \subparagraph free-standing
\makeatletter
\ifx\paragraph\undefined\else
  \let\oldparagraph\paragraph
  \renewcommand{\paragraph}{
    \@ifstar
      \xxxParagraphStar
      \xxxParagraphNoStar
  }
  \newcommand{\xxxParagraphStar}[1]{\oldparagraph*{#1}\mbox{}}
  \newcommand{\xxxParagraphNoStar}[1]{\oldparagraph{#1}\mbox{}}
\fi
\ifx\subparagraph\undefined\else
  \let\oldsubparagraph\subparagraph
  \renewcommand{\subparagraph}{
    \@ifstar
      \xxxSubParagraphStar
      \xxxSubParagraphNoStar
  }
  \newcommand{\xxxSubParagraphStar}[1]{\oldsubparagraph*{#1}\mbox{}}
  \newcommand{\xxxSubParagraphNoStar}[1]{\oldsubparagraph{#1}\mbox{}}
\fi
\makeatother

\usepackage{color}
\usepackage{fancyvrb}
\newcommand{\VerbBar}{|}
\newcommand{\VERB}{\Verb[commandchars=\\\{\}]}
\DefineVerbatimEnvironment{Highlighting}{Verbatim}{commandchars=\\\{\}}
% Add ',fontsize=\small' for more characters per line
\usepackage{framed}
\definecolor{shadecolor}{RGB}{241,243,245}
\newenvironment{Shaded}{\begin{snugshade}}{\end{snugshade}}
\newcommand{\AlertTok}[1]{\textcolor[rgb]{0.68,0.00,0.00}{#1}}
\newcommand{\AnnotationTok}[1]{\textcolor[rgb]{0.37,0.37,0.37}{#1}}
\newcommand{\AttributeTok}[1]{\textcolor[rgb]{0.40,0.45,0.13}{#1}}
\newcommand{\BaseNTok}[1]{\textcolor[rgb]{0.68,0.00,0.00}{#1}}
\newcommand{\BuiltInTok}[1]{\textcolor[rgb]{0.00,0.23,0.31}{#1}}
\newcommand{\CharTok}[1]{\textcolor[rgb]{0.13,0.47,0.30}{#1}}
\newcommand{\CommentTok}[1]{\textcolor[rgb]{0.37,0.37,0.37}{#1}}
\newcommand{\CommentVarTok}[1]{\textcolor[rgb]{0.37,0.37,0.37}{\textit{#1}}}
\newcommand{\ConstantTok}[1]{\textcolor[rgb]{0.56,0.35,0.01}{#1}}
\newcommand{\ControlFlowTok}[1]{\textcolor[rgb]{0.00,0.23,0.31}{\textbf{#1}}}
\newcommand{\DataTypeTok}[1]{\textcolor[rgb]{0.68,0.00,0.00}{#1}}
\newcommand{\DecValTok}[1]{\textcolor[rgb]{0.68,0.00,0.00}{#1}}
\newcommand{\DocumentationTok}[1]{\textcolor[rgb]{0.37,0.37,0.37}{\textit{#1}}}
\newcommand{\ErrorTok}[1]{\textcolor[rgb]{0.68,0.00,0.00}{#1}}
\newcommand{\ExtensionTok}[1]{\textcolor[rgb]{0.00,0.23,0.31}{#1}}
\newcommand{\FloatTok}[1]{\textcolor[rgb]{0.68,0.00,0.00}{#1}}
\newcommand{\FunctionTok}[1]{\textcolor[rgb]{0.28,0.35,0.67}{#1}}
\newcommand{\ImportTok}[1]{\textcolor[rgb]{0.00,0.46,0.62}{#1}}
\newcommand{\InformationTok}[1]{\textcolor[rgb]{0.37,0.37,0.37}{#1}}
\newcommand{\KeywordTok}[1]{\textcolor[rgb]{0.00,0.23,0.31}{\textbf{#1}}}
\newcommand{\NormalTok}[1]{\textcolor[rgb]{0.00,0.23,0.31}{#1}}
\newcommand{\OperatorTok}[1]{\textcolor[rgb]{0.37,0.37,0.37}{#1}}
\newcommand{\OtherTok}[1]{\textcolor[rgb]{0.00,0.23,0.31}{#1}}
\newcommand{\PreprocessorTok}[1]{\textcolor[rgb]{0.68,0.00,0.00}{#1}}
\newcommand{\RegionMarkerTok}[1]{\textcolor[rgb]{0.00,0.23,0.31}{#1}}
\newcommand{\SpecialCharTok}[1]{\textcolor[rgb]{0.37,0.37,0.37}{#1}}
\newcommand{\SpecialStringTok}[1]{\textcolor[rgb]{0.13,0.47,0.30}{#1}}
\newcommand{\StringTok}[1]{\textcolor[rgb]{0.13,0.47,0.30}{#1}}
\newcommand{\VariableTok}[1]{\textcolor[rgb]{0.07,0.07,0.07}{#1}}
\newcommand{\VerbatimStringTok}[1]{\textcolor[rgb]{0.13,0.47,0.30}{#1}}
\newcommand{\WarningTok}[1]{\textcolor[rgb]{0.37,0.37,0.37}{\textit{#1}}}

\providecommand{\tightlist}{%
  \setlength{\itemsep}{0pt}\setlength{\parskip}{0pt}}\usepackage{longtable,booktabs,array}
\usepackage{calc} % for calculating minipage widths
% Correct order of tables after \paragraph or \subparagraph
\usepackage{etoolbox}
\makeatletter
\patchcmd\longtable{\par}{\if@noskipsec\mbox{}\fi\par}{}{}
\makeatother
% Allow footnotes in longtable head/foot
\IfFileExists{footnotehyper.sty}{\usepackage{footnotehyper}}{\usepackage{footnote}}
\makesavenoteenv{longtable}
\usepackage{graphicx}
\makeatletter
\def\maxwidth{\ifdim\Gin@nat@width>\linewidth\linewidth\else\Gin@nat@width\fi}
\def\maxheight{\ifdim\Gin@nat@height>\textheight\textheight\else\Gin@nat@height\fi}
\makeatother
% Scale images if necessary, so that they will not overflow the page
% margins by default, and it is still possible to overwrite the defaults
% using explicit options in \includegraphics[width, height, ...]{}
\setkeys{Gin}{width=\maxwidth,height=\maxheight,keepaspectratio}
% Set default figure placement to htbp
\makeatletter
\def\fps@figure{htbp}
\makeatother

\KOMAoption{captions}{tableheading}
\makeatletter
\@ifpackageloaded{caption}{}{\usepackage{caption}}
\AtBeginDocument{%
\ifdefined\contentsname
  \renewcommand*\contentsname{Table of contents}
\else
  \newcommand\contentsname{Table of contents}
\fi
\ifdefined\listfigurename
  \renewcommand*\listfigurename{List of Figures}
\else
  \newcommand\listfigurename{List of Figures}
\fi
\ifdefined\listtablename
  \renewcommand*\listtablename{List of Tables}
\else
  \newcommand\listtablename{List of Tables}
\fi
\ifdefined\figurename
  \renewcommand*\figurename{Figure}
\else
  \newcommand\figurename{Figure}
\fi
\ifdefined\tablename
  \renewcommand*\tablename{Table}
\else
  \newcommand\tablename{Table}
\fi
}
\@ifpackageloaded{float}{}{\usepackage{float}}
\floatstyle{ruled}
\@ifundefined{c@chapter}{\newfloat{codelisting}{h}{lop}}{\newfloat{codelisting}{h}{lop}[chapter]}
\floatname{codelisting}{Listing}
\newcommand*\listoflistings{\listof{codelisting}{List of Listings}}
\makeatother
\makeatletter
\makeatother
\makeatletter
\@ifpackageloaded{caption}{}{\usepackage{caption}}
\@ifpackageloaded{subcaption}{}{\usepackage{subcaption}}
\makeatother

\ifLuaTeX
  \usepackage{selnolig}  % disable illegal ligatures
\fi
\usepackage{bookmark}

\IfFileExists{xurl.sty}{\usepackage{xurl}}{} % add URL line breaks if available
\urlstyle{same} % disable monospaced font for URLs
\hypersetup{
  pdftitle={HW 6},
  colorlinks=true,
  linkcolor={blue},
  filecolor={Maroon},
  citecolor={Blue},
  urlcolor={Blue},
  pdfcreator={LaTeX via pandoc}}


\title{HW 6}
\author{}
\date{}

\begin{document}
\maketitle


\begin{Shaded}
\begin{Highlighting}[]
\FunctionTok{library}\NormalTok{(tidyverse)}
\end{Highlighting}
\end{Shaded}

\begin{verbatim}
-- Attaching core tidyverse packages ------------------------ tidyverse 2.0.0 --
v dplyr     1.1.4     v readr     2.1.5
v forcats   1.0.0     v stringr   1.5.1
v ggplot2   3.5.1     v tibble    3.2.1
v lubridate 1.9.4     v tidyr     1.3.1
v purrr     1.0.2     
-- Conflicts ------------------------------------------ tidyverse_conflicts() --
x dplyr::filter() masks stats::filter()
x dplyr::lag()    masks stats::lag()
i Use the conflicted package (<http://conflicted.r-lib.org/>) to force all conflicts to become errors
\end{verbatim}

\begin{Shaded}
\begin{Highlighting}[]
\NormalTok{Data }\OtherTok{\textless{}{-}} \FunctionTok{read.csv}\NormalTok{(}\StringTok{"/Users/celestereid/Documents/PLAN372/PLAN372\_2/plan372\_hmks/HW 6/TS3\_Raw\_tree\_data.csv"}\NormalTok{) }\CommentTok{\#importing data}

\NormalTok{Data }\OtherTok{\textless{}{-}}\NormalTok{ Data }\SpecialCharTok{\%\textgreater{}\%}  \CommentTok{\#creating new variabled for the state and city}
  \FunctionTok{mutate}\NormalTok{(}\AttributeTok{State =} \FunctionTok{str\_extract}\NormalTok{(City, }\StringTok{"([:alpha:]+)$"}\NormalTok{), }
         \AttributeTok{Town =} \FunctionTok{str\_replace\_all}\NormalTok{(City, }\StringTok{"([:alpha:]+)$"}\NormalTok{, }\StringTok{" "}\NormalTok{), }\CommentTok{\#getting rid of the state portion to make a city column}
          \AttributeTok{city =} \FunctionTok{str\_replace\_all}\NormalTok{(Town, }\StringTok{"[:punct:]"}\NormalTok{, }\StringTok{" "}\NormalTok{)) }\SpecialCharTok{\%\textgreater{}\%} \CommentTok{\#getting rid of leftover puntiation int he new city column}
  \FunctionTok{select}\NormalTok{( }\SpecialCharTok{{-}} \FunctionTok{c}\NormalTok{(Town)) }\SpecialCharTok{\%\textgreater{}\%}  \CommentTok{\#getting rid of my transition column}
  \FunctionTok{rename}\NormalTok{(}\AttributeTok{City\_State =}\NormalTok{ City, }\CommentTok{\#renaming my variables so they make sense}
         \AttributeTok{City =}\NormalTok{ city)}
      
\CommentTok{\#Now group by City and Count all of them }

\NormalTok{State\_Count }\OtherTok{\textless{}{-}}\NormalTok{ Data }\SpecialCharTok{\%\textgreater{}\%} 
  \FunctionTok{group\_by}\NormalTok{(State) }\SpecialCharTok{\%\textgreater{}\%}  \CommentTok{\#grouping by state}
  \FunctionTok{summarize}\NormalTok{(}\AttributeTok{Count =} \FunctionTok{n}\NormalTok{()) }\CommentTok{\#summarizing to see how many trees were in each state}
\end{Highlighting}
\end{Shaded}

\subsection{Reid Consulting Studios:}\label{reid-consulting-studios}

Reid Consulting Studio's analyst summarized key findings in the
biodiversity and shade cover of trees to prepare analysis of A) the
biodiversity of trees surveyed across states and B) how different tree
genuses correspond to crown cover and potential shade abilities.

Our consulting firm highly values transparency and has included our
representative's coding work, should there be any questions on our
methodologies.

\subsection{Question 1: Sample Sizes by
State}\label{question-1-sample-sizes-by-state}

\begin{Shaded}
\begin{Highlighting}[]
\FunctionTok{ggplot}\NormalTok{()}\SpecialCharTok{+} \CommentTok{\#graphing how many trees were sampled in each state}
  \FunctionTok{geom\_col}\NormalTok{( }\AttributeTok{data =}\NormalTok{ State\_Count, }\FunctionTok{aes}\NormalTok{(State, Count))}\SpecialCharTok{+} \CommentTok{\#pulling data}
  \FunctionTok{labs}\NormalTok{(}
  \AttributeTok{title =} \StringTok{"Sampled Trees in each State"}\NormalTok{,}
  \AttributeTok{caption =} \StringTok{"Figure 1"}\NormalTok{)}\SpecialCharTok{+}
\FunctionTok{ylab}\NormalTok{(}\StringTok{"Tree Count"}\NormalTok{)}\SpecialCharTok{+} \CommentTok{\#adding labels}
  \FunctionTok{theme\_minimal}\NormalTok{()}
\end{Highlighting}
\end{Shaded}

\includegraphics{HW6_files/figure-pdf/unnamed-chunk-2-1.pdf}

Figure 1 displays how many trees were associated with each state in the
dataset. As is apparent in the data, California had the most sample by
around four times any other state. The rests of the states hovered
between around 750 and 1000 samples each.

\subsection{Question 2: Cities in
NC/SC}\label{question-2-cities-in-ncsc}

\begin{Shaded}
\begin{Highlighting}[]
\NormalTok{Carolinas }\OtherTok{\textless{}{-}}\NormalTok{ Data }\SpecialCharTok{\%\textgreater{}\%}  \CommentTok{\#filtering my dataset to just be in NC and SC}
  \FunctionTok{filter}\NormalTok{(State }\SpecialCharTok{\%in\%} \FunctionTok{c}\NormalTok{(}\StringTok{"NC"}\NormalTok{, }\StringTok{"SC"}\NormalTok{))}

\NormalTok{Summary }\OtherTok{\textless{}{-}}\NormalTok{ Carolinas }\SpecialCharTok{\%\textgreater{}\%}  \CommentTok{\#getting just one row for each }
  \FunctionTok{distinct}\NormalTok{(State, City) }

\CommentTok{\#unique(Carolinas$City) checking to make sure there really was only one city per state}

\NormalTok{Summary }\SpecialCharTok{|\textgreater{}} \FunctionTok{as\_tibble}\NormalTok{()}
\end{Highlighting}
\end{Shaded}

\begin{verbatim}
# A tibble: 2 x 2
  State City           
  <chr> <chr>          
1 SC    "Charleston   "
2 NC    "Charlotte   " 
\end{verbatim}

Now that we have a sense of the distribution of our dataset, Reid
Consulting Studios narrowed in on the geographic area most relevant to
our case study, the North Carolina and South Carolina region. In the
above table you can see that only one city was sampled in North Carolina
and one in South Carolina. These cities were Charlotte in NC and
Charleston in South Carolina. This is important to note as we continue
our analysis as there may be bias in the limited distribution of trees
across the state.

\subsection{Question 3: Genera and Species part
1}\label{question-3-genera-and-species-part-1}

\begin{Shaded}
\begin{Highlighting}[]
\CommentTok{\#unique(Carolinas$ScientificName) \#checking that each genus and species is only one word long}

\NormalTok{Carolinas }\OtherTok{\textless{}{-}}\NormalTok{ Carolinas }\SpecialCharTok{\%\textgreater{}\%}  \CommentTok{\#extracting Genus, the first word in the Scientific Name column}
  \FunctionTok{mutate}\NormalTok{(}\AttributeTok{Genus =} \FunctionTok{str\_extract}\NormalTok{(ScientificName, }\StringTok{"(\^{}[:alpha:]+)"}\NormalTok{), }
         \AttributeTok{Genus =} \FunctionTok{str\_to\_lower}\NormalTok{(Genus)) }\CommentTok{\#ensuring that any differences in capitals won\textquotesingle{}t be counted differently}

\NormalTok{Carolina\_genus }\OtherTok{\textless{}{-}}\NormalTok{ Carolinas }\SpecialCharTok{\%\textgreater{}\%} \CommentTok{\#calculating the mean of the crown of each genus}
  \FunctionTok{group\_by}\NormalTok{(Genus) }\SpecialCharTok{\%\textgreater{}\%}  \CommentTok{\#group by genus}
  \FunctionTok{summarize}\NormalTok{( }\AttributeTok{Crown\_size =} \FunctionTok{mean}\NormalTok{(AvgCdia..m.)) }\CommentTok{\#find the mean crown size of each genus}

\CommentTok{\#display the maximum prettier and with the name of the genus from highest to lowest crown diameter}
\NormalTok{Carolina\_genus }\SpecialCharTok{|\textgreater{}} \FunctionTok{as\_tibble}\NormalTok{() }\SpecialCharTok{|\textgreater{}} \FunctionTok{arrange}\NormalTok{(}\SpecialCharTok{{-}}\NormalTok{Crown\_size) }\SpecialCharTok{|\textgreater{}} \FunctionTok{select}\NormalTok{(Genus, Crown\_size)}
\end{Highlighting}
\end{Shaded}

\begin{verbatim}
# A tibble: 20 x 2
   Genus         Crown_size
   <chr>              <dbl>
 1 quercus            13.6 
 2 platanus           12.0 
 3 acer               11.5 
 4 carya              10.5 
 5 betula             10.4 
 6 liquidambar         9.98
 7 ulmus               9.96
 8 celtis              9.45
 9 prunus              9.27
10 pinus               8.66
11 magnolia            7.98
12 gleditsia           7.67
13 pyrus               7.47
14 malus               6.63
15 cornus              6.53
16 lagerstroemia       6.26
17 juniperus           6.01
18 ilex                4.71
19 butia               4.69
20 sabal               3.96
\end{verbatim}

Reid Studio's analyst calulated the crown size of trees in meters. This
was done by genus, so when planting it is important to note that there
may be variation among different species within the genus. Still, it
appears the quercus genus has the largest crown size at 13.6 meters. If
planting for shade cover quercus, platanus, or acer trees may be the
cities best bet.

\subsection{Question 4: Genera and Species part
2}\label{question-4-genera-and-species-part-2}

\begin{Shaded}
\begin{Highlighting}[]
\CommentTok{\#unique(Data$ScientificName) \#getting a sense of the data we have}

\NormalTok{Data }\OtherTok{\textless{}{-}}\NormalTok{ Data }\SpecialCharTok{\%\textgreater{}\%}  \CommentTok{\#extracting genus from the larger dataset }
  \FunctionTok{mutate}\NormalTok{(}\AttributeTok{Genus =} \FunctionTok{str\_extract}\NormalTok{(ScientificName, }\StringTok{"(\^{}[:alpha:]+)"}\NormalTok{),  }\CommentTok{\#extracting the first word aka the genus}
         \AttributeTok{Genus =} \FunctionTok{str\_to\_lower}\NormalTok{(Genus)) }\CommentTok{\#putting everything in lower case to make sure any capitalization errors don\textquotesingle{}t affect outputs}

\NormalTok{sp\_species }\OtherTok{\textless{}{-}}\NormalTok{ Data }\SpecialCharTok{\%\textgreater{}\%}  \CommentTok{\#creating a new object to manipulate species}
   \FunctionTok{mutate}\NormalTok{(}\AttributeTok{ScientificName =} \FunctionTok{str\_replace}\NormalTok{(ScientificName, }\StringTok{"}\SpecialCharTok{\textbackslash{}\textbackslash{}}\StringTok{.$"}\NormalTok{, }\StringTok{""}\NormalTok{),  }\CommentTok{\#getting rid of any periods at the end of the datasets (important for those ending with sp. and cvs. )}
          \AttributeTok{sp\_test =} \FunctionTok{ifelse}\NormalTok{(}\FunctionTok{str\_detect}\NormalTok{(}\FunctionTok{str\_to\_lower}\NormalTok{(ScientificName), }\StringTok{"sp$"}\NormalTok{), }\DecValTok{1}\NormalTok{, }\DecValTok{0}\NormalTok{),  }\CommentTok{\#identifying columns with sp at the end (these were columns with the genus listed but not the species)}
          \AttributeTok{No\_cultivar =} \FunctionTok{str\_replace\_all}\NormalTok{(ScientificName, }\StringTok{"(\textquotesingle{}[:alpha:]+)\textquotesingle{}$"}\NormalTok{, }\StringTok{""}\NormalTok{), }\CommentTok{\#getting rid of all the cultivar names which were designated by being in \textquotesingle{}\textquotesingle{}}
         \AttributeTok{No\_genus =} \FunctionTok{str\_replace\_all}\NormalTok{(No\_cultivar, }\StringTok{"\^{}([:alpha:]+) "}\NormalTok{, }\StringTok{""}\NormalTok{), }\CommentTok{\#getting rid of all the genus names at the front of the column}
         \AttributeTok{No\_x =} \FunctionTok{str\_replace\_all}\NormalTok{(No\_genus, }\StringTok{"\^{}x "}\NormalTok{, }\StringTok{""}\NormalTok{), }\CommentTok{\#getting rid of any unneccisary xs denoting hybrid staus}
         \AttributeTok{No\_var =} \FunctionTok{str\_replace\_all}\NormalTok{(No\_x, }\StringTok{"var. ([:alpha:]+)$"}\NormalTok{, }\StringTok{""}\NormalTok{), }\CommentTok{\#getting rid of any variety indicators at the end of the strings }
         \AttributeTok{No\_subsp =} \FunctionTok{str\_replace\_all}\NormalTok{(No\_var, }\StringTok{"subsp. ([:alpha:]+)$"}\NormalTok{, }\StringTok{""}\NormalTok{), }\CommentTok{\# removing  of subspecies indicators}
         \AttributeTok{No\_cvs =} \FunctionTok{str\_replace\_all}\NormalTok{(No\_subsp, }\StringTok{"cvs$"}\NormalTok{, }\StringTok{""}\NormalTok{), }\CommentTok{\#removing the cvs indicator}
         \AttributeTok{Species =} \FunctionTok{case\_when}\NormalTok{(  }\CommentTok{\#creating the final species column}
\NormalTok{      sp\_test }\SpecialCharTok{==} \DecValTok{1} \SpecialCharTok{\textasciitilde{}} \FunctionTok{paste}\NormalTok{(Genus, }\StringTok{"unspecified"}\NormalTok{), }\CommentTok{\#specifying that when the species is unkown, the column should read the genus name and the unspecified}
\NormalTok{      sp\_test }\SpecialCharTok{==} \DecValTok{0} \SpecialCharTok{\textasciitilde{}}\NormalTok{ No\_cvs, }\CommentTok{\#otherwise using the cleaned species name}
      \ConstantTok{TRUE} \SpecialCharTok{\textasciitilde{}} \StringTok{"NA"}\NormalTok{ ))}

\NormalTok{Identifed\_only }\OtherTok{\textless{}{-}}\NormalTok{ sp\_species }\SpecialCharTok{\%\textgreater{}\%} \CommentTok{\#filtering to just the trees which we know the species of}
  \FunctionTok{filter}\NormalTok{( sp\_test }\SpecialCharTok{==} \DecValTok{0}\NormalTok{) }\SpecialCharTok{\%\textgreater{}\%} 
  \FunctionTok{group\_by}\NormalTok{(Genus) }\SpecialCharTok{\%\textgreater{}\%} \CommentTok{\#grouping by genus}
  \FunctionTok{summarize}\NormalTok{(}\AttributeTok{Species =} \FunctionTok{n\_distinct}\NormalTok{(Species))  }\CommentTok{\#summarizing how many distinct/unique species names existed for each genus}

\NormalTok{Identifed\_only }\SpecialCharTok{|\textgreater{}} \FunctionTok{as\_tibble}\NormalTok{() }\SpecialCharTok{|\textgreater{}} \FunctionTok{arrange}\NormalTok{(}\SpecialCharTok{{-}}\NormalTok{Species) }\CommentTok{\#displaying from most species to least}
\end{Highlighting}
\end{Shaded}

\begin{verbatim}
# A tibble: 85 x 2
   Genus      Species
   <chr>        <int>
 1 pinus           15
 2 quercus         13
 3 fraxinus        11
 4 acer             7
 5 populus          5
 6 prunus           5
 7 ulmus            5
 8 eucalyptus       4
 9 acacia           3
10 celtis           3
# i 75 more rows
\end{verbatim}

Back at the country-wide level (at least among states within our
dataset), we were also interested in biodiversity. Our analyst
calculated the number of species present for each genus around the
country. There are further varieities within some of the genuses, and
there were also a few trees which had only the genus, and not the
species recorded. Our data analyst was and is not an expert on botnary,
nor the naming of tree species and briefly consulted Prof.~El-Khatabbi
on the nomenclature of trees in the data set. Despite initial
consultation, further research and examination of the data determined
that trees ending in sp. meant that the species was not recorded but may
have existed. For instance, within the dataset there existed the Prunus
genus which had multiple species attached to it (cerasifera, carolniana,
etc.), as well as a few ``Prunus sp.'' records. We ommitted the records
containing sp. from our final species count, however our sp\_species
dataset includes these trees as their Genus, followed by the word
``Unspecified'' in the final species column. It was determined that
nation wide the pinus genus of trees has the most variety of species, at
15. This was closely followed by the quercus genus (13 species), and the
fraxinus genus (11 species). Our earlier data modeling suggested that
the quercus genus would also offer the most shade cover. Further
research into which species of quercus would be best for this may be
prudent given the relatviely alrge variation within the genus.

\subsection{Extra Credit}\label{extra-credit}

\begin{Shaded}
\begin{Highlighting}[]
\NormalTok{EC }\OtherTok{\textless{}{-}}\NormalTok{ sp\_species }\SpecialCharTok{\%\textgreater{}\%} 
  \FunctionTok{select}\NormalTok{(Age, Genus, AvgCdia..m.)}

\NormalTok{EC\_agetable }\OtherTok{\textless{}{-}}\NormalTok{ EC }\SpecialCharTok{\%\textgreater{}\%} \CommentTok{\#calculating the mean of the age of each genus}
  \FunctionTok{group\_by}\NormalTok{(Genus) }\SpecialCharTok{\%\textgreater{}\%}  \CommentTok{\#group by genus}
  \FunctionTok{summarize}\NormalTok{( }\AttributeTok{Average\_Age =} \FunctionTok{mean}\NormalTok{(Age)) }\CommentTok{\#find the mean age of each genus}

\NormalTok{  EC\_agetable  }\SpecialCharTok{|\textgreater{}} \FunctionTok{as\_tibble}\NormalTok{() }\SpecialCharTok{|\textgreater{}} \FunctionTok{arrange}\NormalTok{(}\SpecialCharTok{{-}}\NormalTok{Average\_Age) }
\end{Highlighting}
\end{Shaded}

\begin{verbatim}
# A tibble: 85 x 2
   Genus       Average_Age
   <chr>             <dbl>
 1 swietenia          70.0
 2 casuarina          69.9
 3 cocos              55.6
 4 ficus              55.5
 5 calophyllum        46.1
 6 samanea            43.5
 7 cedrus             40.1
 8 robinia            39.6
 9 schinus            39.0
10 cinnamomum         38.6
# i 75 more rows
\end{verbatim}

\begin{Shaded}
\begin{Highlighting}[]
  \FunctionTok{ggplot}\NormalTok{()}\SpecialCharTok{+}
    \FunctionTok{geom\_point}\NormalTok{(}\AttributeTok{data =}\NormalTok{ EC\_agetable, }\FunctionTok{aes}\NormalTok{( Genus, Average\_Age))}\SpecialCharTok{+}
    \FunctionTok{theme\_classic}\NormalTok{()}\SpecialCharTok{+}
      \FunctionTok{labs}\NormalTok{(}
  \AttributeTok{title =} \StringTok{"Average Age of Tree Genuses"}\NormalTok{,}
  \AttributeTok{caption =} \StringTok{"Figure EC{-}1"}\NormalTok{)}\SpecialCharTok{+}
\FunctionTok{ylab}\NormalTok{(}\StringTok{"Average Age (years)"}\NormalTok{)}\SpecialCharTok{+} \CommentTok{\#adding labels}
  \FunctionTok{theme}\NormalTok{(}\AttributeTok{axis.text.x =} \FunctionTok{element\_text}\NormalTok{(}\AttributeTok{angle =} \DecValTok{90}\NormalTok{, }\AttributeTok{size =} \DecValTok{7}\NormalTok{))}
\end{Highlighting}
\end{Shaded}

\includegraphics{HW6_files/figure-pdf/unnamed-chunk-6-1.pdf}

In order to better measure what trees are best suited to shade, and
ensure no confounding variables, our analyst plotted the average age of
each genus in our sample. For the most part these ages hovered around
15-40 years old, however there were also outliers over 60 years old or
near 0. If doing future modeling, we would likely filter to exclude
trees whose ages are less than 5, as they will not have had long enough
to grow to have much of a crown at all. Because of the variations in
this data, it is possible that age had an impact on the average canopy
size of diferent genus in the previous question. Still, the quercus
genus which we had found to have the highest average canopy size has an
average age of about 29 years, which is in the mid-range of our data.

\begin{Shaded}
\begin{Highlighting}[]
\NormalTok{EC\_fast }\OtherTok{\textless{}{-}}\NormalTok{ EC }\SpecialCharTok{\%\textgreater{}\%} 
  \FunctionTok{group\_by}\NormalTok{(Genus) }\SpecialCharTok{\%\textgreater{}\%}  \CommentTok{\#group by genus}
  \FunctionTok{summarize}\NormalTok{( }\AttributeTok{Average\_Age =} \FunctionTok{mean}\NormalTok{(Age), }
             \AttributeTok{Avg\_Crown =} \FunctionTok{mean}\NormalTok{(AvgCdia..m.)) }\SpecialCharTok{\%\textgreater{}\%} 
  \FunctionTok{mutate}\NormalTok{( }\AttributeTok{Average\_Age\_color  =} \FunctionTok{case\_when}\NormalTok{(  }\CommentTok{\#creating the final species column}
\NormalTok{      Average\_Age }\SpecialCharTok{\textless{}=} \DecValTok{20} \SpecialCharTok{\textasciitilde{}} \StringTok{"Less than 20"}\NormalTok{, }\CommentTok{\#specifying that when the species is unkown, the column should read the genus name and the unspecified}
\NormalTok{      Average\_Age }\SpecialCharTok{\textless{}=} \DecValTok{40} \SpecialCharTok{\textasciitilde{}} \StringTok{"20{-}40"}\NormalTok{, }
\NormalTok{      Average\_Age }\SpecialCharTok{\textgreater{}=}\DecValTok{40} \SpecialCharTok{\textasciitilde{}} \StringTok{"Over 40"}\NormalTok{, }
      \ConstantTok{TRUE} \SpecialCharTok{\textasciitilde{}} \StringTok{"NA"}\NormalTok{ ))}


\FunctionTok{ggplot}\NormalTok{(EC\_fast, }\FunctionTok{aes}\NormalTok{( Genus, Avg\_Crown, }\AttributeTok{color =}\NormalTok{ Average\_Age\_color))}\SpecialCharTok{+}
  \FunctionTok{geom\_point}\NormalTok{()}\SpecialCharTok{+}
  \FunctionTok{labs}\NormalTok{(}
  \AttributeTok{title =} \StringTok{"Genus Crown Size with Age Component"}\NormalTok{,}
  \AttributeTok{caption =} \StringTok{"Figure EC{-}2"}\NormalTok{)}\SpecialCharTok{+}
\FunctionTok{ylab}\NormalTok{(}\StringTok{"Average Crown Diameter (m)"}\NormalTok{)}\SpecialCharTok{+} \CommentTok{\#adding labels}
  \FunctionTok{theme\_minimal}\NormalTok{()}\SpecialCharTok{+}
  \FunctionTok{theme}\NormalTok{(}\AttributeTok{axis.text.x =} \FunctionTok{element\_text}\NormalTok{(}\AttributeTok{angle =} \DecValTok{90}\NormalTok{, }\AttributeTok{size =} \DecValTok{7}\NormalTok{))}
\end{Highlighting}
\end{Shaded}

\includegraphics{HW6_files/figure-pdf/unnamed-chunk-7-1.pdf}

\begin{Shaded}
\begin{Highlighting}[]
\NormalTok{EC\_final }\OtherTok{\textless{}{-}}\NormalTok{ EC\_fast }\SpecialCharTok{\%\textgreater{}\%} 
  \FunctionTok{filter}\NormalTok{(Average\_Age }\SpecialCharTok{\textless{}=} \DecValTok{20}\NormalTok{) }\SpecialCharTok{\%\textgreater{}\%} 
  \FunctionTok{select}\NormalTok{(}\SpecialCharTok{{-}}\NormalTok{Average\_Age\_color)}


 

\NormalTok{EC\_final }\SpecialCharTok{|\textgreater{}} \FunctionTok{as\_tibble}\NormalTok{() }\SpecialCharTok{|\textgreater{}} \FunctionTok{arrange}\NormalTok{(}\SpecialCharTok{{-}}\NormalTok{Avg\_Crown)}
\end{Highlighting}
\end{Shaded}

\begin{verbatim}
# A tibble: 21 x 3
   Genus        Average_Age Avg_Crown
   <chr>              <dbl>     <dbl>
 1 zelkova             18.1     12.3 
 2 morus               17.9     12.3 
 3 prosopis            17        9.27
 4 aesculus            -1        9.12
 5 parkinsonia         18.2      7.95
 6 pistacia            19.2      7.40
 7 koelreuteria        19.2      7.23
 8 cercis              -1        7.18
 9 prunus              20.0      7.12
10 triadica            14.2      7.11
# i 11 more rows
\end{verbatim}

When factoring in Average ages under 20 years old, the genus with the
largest average crown is zelkova. In other words, if the city want a
short term shade return on investment, zelkova or morus would be the way
to go.




\end{document}
